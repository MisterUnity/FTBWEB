Draggable屬性說明


allowAnyClick: boolean // 默认false,设为true非左键可实现点击拖拽
axis: string // 'x':x轴方向拖拽、'y':y轴方向拖拽、'none':禁止拖拽
bounds: { left: number, top: number, right: number, bottom: number } | string 
    // 限定移动的边界,接受值:
    //(1)'parent':在移动元素的offsetParent范围内
    //(2)一个选择器,在指定的Dom节点内
    //(3){ left: number, top: number, right: number, bottom: number }对象,限定每个方向可以移动的距离
cancel:制定给一个选择器组织drag初始化,例如'.body'
defaultClassName:string // 拖拽ui类名,默认'react-draggable'
drfaultClassNameDragging:string // 正在拖拽ui类名,默认'eact-draggable-dragging'
defaultClassNameDragged:string //拖拽后的类名,默认'react-draggable-dragged'
defaultPosition:{ x: number, y: number } // 起始x和y的位置
disabled:boolean // true禁止拖拽任何元素
grid:[number, number] // 正在拖拽的网格范围
handle:string  // 初始拖拽的的选择器'.handle'
offsetParent:HTMLElement // 拖拽的offsetParent
onMouseDown: (e: MouseEvent) => void // 鼠标按下的回调
onStart: DraggableEventHandler // 开始拖拽的回调
onDrag:DraggableEventHandler // 拖拽时的回调
onStop:DraggableEventHandler // 拖拽结束的回调
position: {x: number, y: number} // 控制元素的位置
positionOffset: {x: number | string, y: number | string} // 相对于起始位置的偏移
scale:number // 定义拖拽元素的缩放

作者:笪笪
链接:https://juejin.cn/post/6844903878148751374
来源:稀土掘金
著作权归作者所有。商业转载请联系作者获得授权,非商业转载请注明出处。